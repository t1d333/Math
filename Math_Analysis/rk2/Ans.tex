\documentclass[12pt, a4paper]{article}
\usepackage[T2A]{fontenc}
\usepackage[utf8]{inputenc}
\usepackage[russian,english]{babel}
\usepackage{amsmath}
\usepackage{amsthm}
\usepackage{amsfonts}
\newtheorem{thm}{Теорема}
\linespread{1.3}

\title{Ответы к РК2}
\author{Кирилл Киселев} 
\date{Октябрь 2022}
\begin{document}
\maketitle
\newpage

\section{Теорема о непрерывности собственного интеграла, зависящего от параметра.}
\begin{thm}
О непрерывности собственного интеграла зависящего от параметра
\end{thm}
Пусть функция $f(x, y)$ определна на прямоугольнике $[a, b] \times [c, d]$ и непрерывна по совокупности переменных.
Тогда интеграл, зависящий от параметра
    \[
        I(y) = \int\limits_a^b f(x, y)dx
    \]
является непрерывной функцией на отрезке $[c, d]$

\section{Правило Лейбница для вычисления производной собственного интеграла, зависящего от параметра}
\begin{thm}
    Правило Лейбница
\end{thm}
Пусть $f$ определена и непрерывна на прямоугольнике $[a, b] \times [c, d]$ и
непрерывна по $x$ на отрезке $[a, b]$ при каждом фиксированном $y \in [c, d]$.
Также потребуем, чтобы функция $f'_y(x, y)$ существовала и была непрерывной на
всем прямоугольнике $[a, b] \times [c, d]$. Тогда
\[
    I(y) = \int\limits_a^b f(x, y)dx
\]
Является дифференцируемой функцией параметра $y$ на всем отрезке $[a, b]$.
И производная $I'(y)$ может быть найдена по правилу Лейбница:
\[
    I'(y) = \int\limits_a^b f'_y(x, y)dx
\]
т.е. с помощью дифферования по параметру под знаком дифференциала.

\section{Интегрирование по параметру собственного интеграла, зависящего от параметра.}
\begin{thm}
    Об интегрировании собственного интеграла, зависящего от параметра
\end{thm}
Пусть функция $f(x, y)$ непрерывна по совокупности переменных на прямоугольнике 
$[a, b] \times [c, d]$. Тогда
\[
    \int\limits_c^d dy \int\limits_a^b f(x, y)dx = \int\limits_a^b dx \int\limits_c^d f(x, y)dy  
    \]
\section{Дифференцирование интеграла по параметру в случае, когда и пределы интегрирования зависит от параметра.}
Рассмотрим вопрос о дифференцировании интеграла, зависящего от параметра в случае
, когда пределы интегрирования зависят от зависят от параметра. Пусть функция
$f(x, y)$ непрерывна в точках прямоугольника $[a, b] \times [c, d]$, и пусть
кривые $x = \alpha(y)$ и $x = \beta(y)$, c $\leq$ y $\leq$ d, целиком лежат
в этом прямоугольнике. Рассмотрим интеграл $I(y) = \int\limits_{\alpha(y)}^{\beta(y)}f(x, y)dx$ 
в виде $I(y, u, v)$ = $\int\limits_u^v f(x, y)dx$, где $u = \alpha(y), v = \beta(y)$ 
Тогда по правилу дифференцирования сложной функции 
\[
    \frac{d}{dy}I(y, u, v) = \frac{\partial I}{\partial y} + 
    \frac{\partial I}{\partial u} * \frac{\partial u}{\partial y} + 
    \frac{\partial I}{\partial v} *  \frac{\partial v}{\partial y}  = 
    \int\limits_{\alpha(y)}^{\beta(y)}f'_y(x, y)dx - f(\alpha(y), y)\alpha'(y) + 
    f(\beta(y), y)\beta'(y).
\]
Т.о. получим требуемую формулу
\[
    \frac{d}{dy}\int\limits_{\alpha(y)}^{\beta(y)}f(x, y)dx = 
    \int\limits_{\alpha(y)}^{\beta(y)}f'_y(x, y)dx - f(\alpha(y), y)\alpha'(y) + 
    f(\beta(y), y)\beta'(y).
\]

\section{Определение равномерной сходимости несобственного интеграла, зависящего от параметра.}
Пусть функция $f(x, y)$ определена при $x \geq a$ и $y \in Y$.
Пусть при каждом фиксированном $y \in Y$ существует несобственный интеграл:
\[
    I(y) = \int\limits_a^{\infty} f(x, y)dx
\]
Это означает, что существует конечный предел $\lim\limits_{A \to +\infty}
\int\limits_a^A f(x, y)dx$.
Если обозначить $F(y, A) = \int\limits_a^A f(x, y)dx$, то 
$\lim\limits_{A \to +\infty}F(y, A) = \int\limits_a^{\infty} f(x, y)dx$.
Если при $A \to +\infty$ функция $F(y, A)$ стремится к своему пределу равномерно
относительно $y$, то говорят, что несобственный интеграл $\int\limits_a^{\infty} f(x, y)dx$
сходится равномерно относительно $y \in Y$.
Это означает, что

    $\forall$ $\varepsilon$ > 0 $\exists A_0$ = $A_0(\varepsilon)$ такое, что 
при любом  $A \geq A_0$ и при любом $y \in Y$ выполняется неравенство
    $|\int\limits_a^{\infty} f(x, y)dx - \int\limits_a^A f(x, y)dx |$  = 
    $|\int\limits_A^{\infty} f(x, y)dx|$ < $\varepsilon$

\section{Критерий Коши равномерной сходимости несобственного интеграла, зависящего от параметра.}
Пусть функция $f(x, y)$ определена при $x \geq a$ и $y \in Y$.
Интеграл $I(y) = \int\limits_a^{\infty} f(x, y)dx$ сходится равномерно относительно
$y \in Y$ тогда и только тогда, когда $\forall$ $\varepsilon$ > 0 $\exists 
A_0$ = $A_0(\varepsilon), A_0 \geq a$ такое, что для любых $A_1$ и $A_2$, 
$A_1 \geq A_0$, $A_2 \geq A_0$ выполняется неравенство 
$|\int\limits_a^{A_1} f(x, y)dx - \int\limits_a^{A_2} f(x, y)dx |$  = 
$|\int\limits_{A_1}^{A_2} f(x, y)dx|$ < $\varepsilon$

\section{Признак Вейерштрасса равномерной сходимости интеграла, зависящего от параметра.}
Пусть функция $f(x, y)$ определена при $x \geq a$ и $y \in Y$.
Пусть также выполняется неравенство $|f(x, y)|$ $\leq \varphi(x)$($\varphi$
- мажоранта) для всех $y \in Y$. Тогда если интеграл $\int\limits_a^{\infty}
\varphi(x)dx$ сходится, то интеграл $\int\limits_a^{\infty}f(x, y)dx$
сходится равномерно относительно $y \in Y$

\section{Примеры равномерно и неравномерно сходящегося несобственного интеграла, зависящего от параметра.}
Пример 1.
\newline
Проверим, что интеграл $\int\limits_0^{\infty}\frac{cos(\alpha x)}{1 + x^2}dx$
сходится равномерно относительно $\alpha \in \mathbb{R}$.
Т.к. при всех $x$ $\geq$ $0$ и при всех $\alpha \in \mathbb{R}$ выполняется
неравенство $|\frac{cos(\alpha x)}{1 + x^2}|$ $\leq$ $\frac{1}{1 + x^2}$,
и интеграл $\int\limits_0^{\infty}\frac{1}{1 + x^2}$ сходится, то исходный интеграл
сходится равномерно относительно $\alpha \in \mathbb{R}$ по теореме Вейерштрасса.
\newline
Пример 2.
\newline
$\int\limits_0^{\infty}\frac{cos(\alpha x)}{1 + \alpha^2 x^2}$, $\alpha \in
(0, +\infty)$. Пусть выполнен критерий Коши сходимости интеграла, зависящего
от параметра. Тогда $\forall$ $\varepsilon$ > 0 $\exists A_0$ > 0 такое, что
если $A_1 > A_0$, $A_2 > A_0$, то $| \int\limits_{A_1}^{A_2}\frac{cos(\alpha x)}
{1 + \alpha ^ 2 x^2}dx|$ < $\varepsilon$ при любом $\alpha \in 
(0, + \infty)$. Преобразуем последний интеграл: $\int\limits_{A_1}^{A_2}\frac{cos(\alpha x)}
{1 + \alpha ^ 2 x^2}dx$ = (замена: $\alpha x = t$) = $\int\limits_{\alpha A_1}^
{\alpha A_2}\frac{1}{\alpha}\frac{cos(t)}{1 + t^2}dt$. Пусть $A_1$ = $2\pi n$,
$A_2$ = $2\pi n + \frac{\pi n}{4}$, $\alpha = \frac{1}{n}$. Тогда имеем такое
неравенство: 

$$| n\int\limits_{2\pi}^{2\pi + \frac{\pi}{4}}\frac{cost}{1 + t^2}dt| < \varepsilon$$

Так как на отрезке $[2\pi, 2\pi + \frac{\pi}{4}]$ выполняется неравенство
$cos(t) $ $ \geq$ $ \frac{\sqrt{2}}{2}$, то написанное неравенство не 
выполняется при достаточно большом значении $n$(ни при каком $\varepsilon$).
Равномерной сходимости нет.

\section{Теорема о непрерывности несобственного интеграла, зависящего от параметра.}
\begin{thm}
    О непрерывности несобсвтенного интеграла, зависящего от параметра
\end{thm}
Пусть функция $f(x, y)$ определена и непрерывна по совокупности переменных
на множестве $[a, +\infty] \times [c, d]$. Пусть далее интеграл
$\int\limits_a^{\infty}f(x, y)dx$ сходится равномерно относительно $y \in [c, d]$.
Тогда этот интеграл является непрерывной функцией на отрезке $[c, d]$

\section{Теорема об интегрировании несобственного интеграла, зависящего от параметра.}
Пусть функция $f(x, y)$ непрерывна по совокупности переменных, как функция
двух переменных на множестве $[a, +\infty] \times [c, d]$. Тогда если
интеграл $\int\limits_a^{\infty}f(x, y)dx$ сходится равномерно относительно
$y \in [c, d]$, то выполняется равенство
$$\int\limits_c^d dy \int\limits_a^{+\infty}f(x, y)dx = 
\int\limits_a^{+\infty} dx \int\limits_c^d f(x, y)dy $$

\section{Теорема о дифференцировании несобственного интеграла, зависящего от параметра.}
Пусть на множестве $[a, +\infty] \times [c, d]$ непрерывны функции
$f(x, y)$ и $f'_y(x, y)$. Пусть, далее, $\int\limits_a^{\infty}f(x, y)dx$
сходится при каждом $y \in [c, d]$, а интеграл
$\int\limits_a^{\infty}f'_y(x, y)dx$ сходится равномерно
относительно $y \in [c, d]$. Тогда выполняется равенство
$$\frac{d}{dy}\int\limits_a^{\infty}f(x, y)dx  = 
\int\limits_a^{\infty}f'_y(x, y)dx$$

Последнее равенство подразумевает и существование производной из левой части.
\section{Бесконечномерное евклидово пространство и норма в таком пространстве.}

\section{Ортогональные и ортонормированные системы в бесконечномерном евклидовом пространстве.}
Пусть $E$ - евклидово пространство. Последовательность $\{\varphi_n\}_{n=1}^{\infty}$
не нулевых элементов $E$ называется ортогональной системой, если
$(\varphi_k, \varphi_l) = 0$ при $k \ne l$. Ортогональные системы могут существовать
только в бесконечномерных пространствах. Ортогональная система называется
ортогонормированной, если нормы всех векторов этой системы равны 1. Из
ортогональной системы $\{\varphi_n\}_{n=1}^{\infty}$ нетрудно получить
ортогонормированную, перейдя к системе $\{\frac{\varphi_n}{\| \varphi_n \|}\}_{n=1}^{\infty}$

\section{Коэффициенты Фурье и ряд Фурье по ортогональной системе.}
Пусть $f \subset E$, по аналогии с конечномерным случаем вычислим коэффициенты
Фурье элемента $f$ по ортогональной системе $\{\varphi_n\}_{n=1}^{\infty}$
$$c_n = \frac{(f, \varphi_n)}{{\| \varphi_n \|}^2}, \quad n=1,2,3 ...$$
Составим ряд Фурье элемента $f$ по указанной ортогональной системе:
$$ f \sim c_1 \varphi_1 + c_2 \varphi_2 + ... + c_n \varphi_n + ... =
\sum\limits_{n=1}^{\infty}c_n \varphi_n$$

\section{Теорема о минимальном свойстве коэффициентов Фурье.}
\begin{thm}
    о минимальном свойстве коэффициентов Фурье.
\end{thm}
Пусть $E$ - евклидово пространство. Последовательность $\{\varphi_n\}_{n=1}^{\infty}$
- ортогональная система в этом пространстве, и пусть $f \in E$.
Тогда минимальное значение величины $\| f - \sum\limits_{n=1}^{\infty}\alpha_n \varphi_n\|$,
где $\alpha_1, \alpha_2, ... , \alpha_n$ - произвольные вещественные
числа, достигается при $\alpha_k = c_k$, где $c_k = \frac{(f, \varphi_n)}{{\| \varphi_n \|}^2}$
- коэффициенты Фурье элемента $f$ по ортогональной системе $\{\varphi_n\}_{n=1}^{\infty}$

\section{Замечание к теореме о минимальном свойстве коэффициентов Фурье.}
Замечание
\newline
По ходу доказательства теоремы установлено такое равенство
$$\|f - \sum\limits_{k=1}^{n}c_k \varphi_k\| = \|f\|^2 - \sum\limits_{k=1}^{n}c_k ^ 2 \| \varphi_k\|^2$$
Здесь левая часть неорицательна, и, следовательно, $$\|f\|^2 \geq \sum\limits_{k=1}^{n}c_k ^ 2 \| \varphi_k\|^2$$
В последнем неравенстве можно перейти к пределу при $n \to \infty$; в результате
получаем неравенство
$$\|f\|^2 \geq \sum\limits_{k=1}^{\infty}c_k ^ 2 \| \varphi_k\|^2$$,
которое называется неравенством Бесселя.
\newline
Если $\forall f \in E$ выполняется равенство 
$$\|f\|^2 = \sum\limits_{k=1}^{\infty}c_k ^ 2 \| \varphi_k\|^2$$,
то соответствуящая ортогональная система называется замкнутой. При этом
последнее равенство называется равенством Парсеваля.
\newline
Из равенства
$$\|f - \sum\limits_{k=1}^{n}c_k \varphi_k\| = \|f\|^2 - \sum\limits_{k=1}^{n}c_k ^ 2 \| \varphi_k\|^2$$
следует, что ряд Фурье любого элемента $f \in E$ сходится к $f$ по норме пространства
$E$ тогда и только тогда, когда ортогональная система $\{\varphi_n\}_{n=1}^{\infty}$
является замкнутой.

\section{Неравенство Бесселя и равенство Парсеваля. Замкнутость ортогональной системы.}
Неравенство Бесселя.
$$\|f\|^2 \geq \sum\limits_{k=1}^{\infty}c_k ^ 2 \| \varphi_k\|^2$$
Равенство Парсеваля
$$\|f\|^2 = \sum\limits_{k=1}^{\infty}c_k ^ 2 \| \varphi_k\|^2$$
Если $\forall f \in E$ выполняется равенство 
$$\|f\|^2 = \sum\limits_{k=1}^{\infty}c_k ^ 2 \| \varphi_k\|^2$$
то соответствуящая ортогональная система называется замкнутой. 
\section{Замкнутость ортогональной системы и сходимость соответствующего ряда Фурье}
Если $\forall f \in E$ выполняется равенство 
$$\|f\|^2 = \sum\limits_{k=1}^{\infty}c_k ^ 2 \| \varphi_k\|^2$$,
то соответствуящая ортогональная система называется замкнутой. При этом
последнее равенство называется равенством Парсеваля.
\newline
Из равенства
$$\|f - \sum\limits_{k=1}^{n}c_k \varphi_k\| = \|f\|^2 - \sum\limits_{k=1}^{n}c_k ^ 2 \| \varphi_k\|^2$$
следует, что ряд Фурье любого элемента $f \in E$ сходится к $f$ по норме пространства
$E$ тогда и только тогда, когда ортогональная система $\{\varphi_n\}_{n=1}^{\infty}$
является замкнутой.
\section{Пространство непрерывных функций и скалярное произведение в этом пространстве.}
Рассмотрим множество всех непрерывных функций на отрезке $[a, b]$.
На этом множестве (которое очевидным образом превращается в линейное пространство)
введем скалярное произведение по формуле
$$(f, g) = \int\limits_a^b f(x)g(x)dx$$
При этом выполняются все аксиомы скалярного умножения
\begin{enumerate}
    \item{$(f, g) = (g, f)$}
    \item{$(\alpha f, g) = \alpha(f, g)$}
    \item{$(f_1 + f_2, g) = (f_1, g) + (f_2, g)$}
    \item{$(f, f) \geq 0, \ (f, f) = 0 \Longrightarrow f = \overrightarrow{0} $}
\end{enumerate}
Полученное пространство обозначается $C_2[a, b]$
\section{Тригонометрическая система на отрезке $[-\pi, \pi]$. Проверка ортогональности и вычисление норм.}
Важнейшим примерос ортогональной системы в этос пространстве является 
тригонометрическая система:
$$\{1,\ cos(\frac{2\pi n x}{b-a}),\ sin(\frac{2\pi n x}{b-a})\}, \ n = 1, 2, 3, ...$$
Подробнее разберем случай отрезка $[-\pi, \pi]$. Тригонометрическая 
система на этом отрезке имеет вид: $1, \ cos(x), \ sin(x), \ ... ,cos(nx),\ sin(nx), ...$
Проверим ортогональность этой системы:
$$(1,\ cos(nx)) \ = \int\limits_{-\pi}^{\pi}cos(nx)dx \ = 
\frac{sin(nx)}{n}\bigg|_{-\pi}^{\pi} \ = 0$$

$$(1,\ sin(nx)) \ = \int\limits_{-\pi}^{\pi}sin(nx)dx \ = 
- \frac{cos(nx)}{n}\bigg|_{-\pi}^{\pi} \ = 0$$

$$(cos(mx),\ cos(nx)) \ = \int\limits_{-\pi}^{\pi}cos(mx)cos(nx)dx \ = 
\frac{1}{2} \int\limits_{-\pi}^{\pi}(cos((m + n)x) + cos((m - n)x))dx \ =
$$

$$\ = \frac{1}{2}(\frac{sin((m + n)x)}{m + n} + \frac{sin((m - n)x)}{m -n})\bigg|_{-\pi}^{\pi} = 0,\ m \ne n,\ m,\ n \ = \ 1, 2, ...$$

Аналогично можно проверить равенства $\int\limits_{-\pi}^{\pi}cos(mx)sin(nx)dx \ = 0$,
$\int\limits_{-\pi}^{\pi}sin(mx)sin(nx)dx \ = 0$, в последнем интеграле $m  \ne n, \ m, n = 1, 2, ...$
\newline
Вычислим нормы элементов тригонометрической системы :
$$
    \|1\|^2 = \int\limits_{-\pi}^{\pi}1 * 1 dx = 2\pi
$$
$$
    \| cos(nx) \|^2 = \int\limits_{-\pi}^{\pi}cos^2(nx)dx = 
    \int\limits_{-\pi}^{\pi}\frac{1 + cos(2nx)}{2}dx = 
    \frac{1}{2}(x  + \frac{sin(2nx)}{2n})\bigg|_{-\pi}^{\pi} = \pi
$$

$$
    \| sin(nx) \|^2 = \int\limits_{-\pi}^{\pi}sin^2(nx)dx = \ldots = 
    \pi
$$

\section{Равенство Парсеваля для тригонометрической системы на отрезке $[-\pi, \pi]$.}
Равенство Парсеваля в случае отрезка $[-\pi, \pi]$ выглаядит так:
$$
\frac{1}{\pi} \int\limits_{-\pi}^{\pi} f^2(x)dx = \frac{a^2_0}{2} + 
\sum\limits_{n=1}^{\infty}(a^2_n + b^2_n) 
$$

\section{Сходимость в среднем квадратичном для тригонометрического ряда; определение сходимости в среднем квадратичном для последовательности функций.}
Сходимость последовательности функций $f_n(x)$, заданных на отрезке
$[a, b]$ к функции $f(x)$ по норме пространства $C_2[a, b]$ означает,
что 
$$\lim\limits_{n \to \infty}\int\limits_a^b(f_n(x) - f(x))^2dx = \ 0$$
Все функции предполагаются непрерывными. Такая сходимость называется
сходимостью в среднем квадратичном.
\newline
Для тригонометрического радя Фурье она обозначает, что
$$\lim\limits_{n \to \infty}\int\limits_{-\pi}^{\pi}(f(x) - (\frac{a^2_0}{2} + 
\sum\limits_{k=1}^{\infty}(a_k cos(kx) + b_k sin(kx)) )^2dx = \ 0$$

\section{Условия Дирихле и теорема Дирихле.}
Если функция одновременно кусочно непрерывнаи кусочно монотонна на 
некотором отрезке, то говорят, что на этом отрезке функция удовлетворяет условиям
Дирихле.
\begin{thm}
    Дирихле
\end{thm}
Пусть функция $f: [-\pi, \pi] \to \mathbb{R}$ кусочно непрерывна и кусочно
монотонна на отрезке $[-\pi, \pi]$. Тогда ряд Фурье этой функции по
тригонометрической системе сходится в каждой точке $x \in \mathbb{R}$, и его
сумма $S(x)$ является $2\pi$ периодической функцией.
\newline
Если $x \in (-\pi, \pi)$, то $S(x)$ = $\frac{1}{2}(f(x - 0) + f(x + 0))$,
где $f(x - 0) \ = \lim\limits_{t \to x - 0}f(t)$,
$f(x + 0) \ = \lim\limits_{t \to x + 0}f(t)$; в частности, в точке $x$
непрерывности функции $f$ сумма ряда $S(x) = \ f(x)$. Далее,
$S(-\pi) = S(\pi) = \frac{1}{2}(f(-\pi + 0) - f(\pi - 0))$

\section{Применение теоремы Дирихле для изучения поведения неполных рядов Фурье.}
Пусть, например, дана функция $f: [0, \pi] \to \ \mathbb{R}$, и этой
функции поставлен в соответствие ряд Фурье по синусам:
$$ 
f \sim \sum\limits_{n=1}^{\infty}b_n sin(nx), \text{где} \ b_n = 
\frac{2}{\pi}\int\limits_{0}^{\pi}f(t)sin(nt)dt
$$
Продолжим функцию $f(x)$ на полуинтервал $[-\pi, 0)$, положив
$f(-x) = -f(x)$. Продолженную таким образом на веь отрезок
$[-\pi, \pi]$ также будем обозначать $f(x)$. Для функции
$f(x)$ ряд Фурье будет иметь такое же вид, как и для
исходной функции, и применение теоремы Дирихле здесь
возможно (если только продолженная функция удовлетворяет требованиям этой теоремы).
То же самое верно и для разложения в неполный ряд по косинусам.
$$ 
f \sim \frac{a_0}{2} + \sum\limits_{n=1}^{\infty}a_n cos(nx), \text{где} \ a_n = 
\frac{2}{\pi}\int\limits_{0}^{\pi}f(t)cos(nt)dt
$$
\section{Тригонометрический ряд Фурье на отрезке}
Если дана функция $f: [-l, l] \to \ \mathbb{R}$, где $l > 0$, то
мы можем рассмотреть тригонометрическую систему:
$$
1, \ cos(\frac{\pi x}{l}), \ sin(\frac{\pi x}{l}), \ldots, 
cos(\frac{n\pi x}{l}), \ sin(n\frac{\pi x}{l}), \ldots ,
$$
которая ортогональна на отрезке $[-l, l]$. Ряд Фурье функции $f$ по такой системе
имеет вид:
$$
f \sim \frac{a_0}{2} + \sum\limits_{n=1}^{\infty}a_n cos(\frac{n \pi x}{l})
+ sin(\frac{n \pi x}{l}) , \ \text{где}\ 
$$
$$
a_n = \ \frac{1}{l}\int\limits_{-l}^{l}f(t)cos(\frac{n\pi t}{l})dt
$$
$$
b_n = \ \frac{1}{l}\int\limits_{-l}^{l}f(t)sin(\frac{n\pi t}{l})dt
$$
Ортонормированная тригонометрическая система на отрезке $[-l, l]$ такова:
$$
\frac{1}{\sqrt{2l}},\ \frac{1}{\sqrt{l}}cos(\frac{\pi x }{l}),\ 
\frac{1}{\sqrt{l}}sin(\frac{\pi x }{l}),\  \ldots \ , \frac{1}{\sqrt{l}}cos(\frac{\pi x n}{l}),\ 
\frac{1}{\sqrt{l}}sin(\frac{\pi x n}{l}), \ \ldots \ ,  
$$
\end{document}
