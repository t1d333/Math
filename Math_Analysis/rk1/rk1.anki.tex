% -*- coding: utf-8-unix -*-
%%%%%%%%%%%%%%%%%%%%%%%%%%%%%%%%%%%%%%%%%%%%%%%%%%%%
% The first part of the header needs to be copied
%       into the note options in Anki.
%%%%%%%%%%%%%%%%%%%%%%%%%%%%%%%%%%%%%%%%%%%%%%%%%%%%

% layout in Anki:
\documentclass[12pt, a4paper]{article}
\usepackage[a4paper]{geometry}
\geometry{paperwidth=.5\paperwidth,paperheight=100in,left=2em,right=2em,bottom=1em,top=2em}
\pagestyle{empty}
\setlength{\parindent}{0in}

% hyphenation:
%% \usepackage[ngerman]{babel}

% encoding:
\usepackage[T2A]{fontenc}
\usepackage[utf8]{inputenc}
\usepackage[russian,english]{babel}

% packages:
\usepackage{amsmath}
\usepackage{amsthm}
\usepackage{amsfonts}
\newtheorem{thm}{Теорема}

%%%%%%%%%%%%%%%%%%%%%%%%%%%%%%%%%%%%%%%%%%%%%%%%%%%%
% Following part of header NOT to be copied into
%            the note options in Anki.
%          ! Anki will throw an errow !
%%%%%%%%%%%%%%%%%%%%%%%%%%%%%%%%%%%%%%%%%%%%%%%%%%%%%
%
%  pdf layout:
%
\geometry{paperheight=74.25mm}
\usepackage{pgfpages}
\pagestyle{empty}
\pgfpagesuselayout{8 on 1}[a4paper,border shrink=0cm]
\makeatletter
\@tempcnta=1\relax
\loop\ifnum\@tempcnta<9\relax
\pgf@pset{\the\@tempcnta}{bordercode}{\pgfusepath{stroke}}
\advance\@tempcnta by 1\relax
\repeat
\makeatother
%
%  notes, fields, tags:
%
\newcommand{\xfield}[1]{
        #1\par
        \vfill
        {\tiny\texttt{\parbox[t]{\textwidth}{\localtag\\\globaltag\hfill\uuid}}}
        \newpage}
\newenvironment{field}{}{\newpage}
\newif\ifnote
\newenvironment{note}{\notetrue}{\notefalse}
\newcommand{\localtag}{}
\newcommand{\globaltag}{}
\newcommand{\uuid}{}
\newcommand{\tags}[1]{
    \ifnote
        \renewcommand{\localtag}{#1}
    \else
        \renewcommand{\globaltag}{#1}
    \fi
    }
\newcommand{\xplain}[1]{\renewcommand{\uuid}{#1}}
%
%%%%%%%%%%%%%%%%%%%%%%%%%%%%%%%%%%%%%%%%%%%%%%%%%%%%
% The following line again needs to be copied
% into Anki:
\begin{document}
%%%%%%%%%%%%%%%%%%%%%%%%%%%%%%%%%%%%%%%%%%%%%%%%%%%%

\tags{rk1}


\begin{note}
\begin{field}
Числовые ряды: определение, частичные суммы и сходимость.
\end{field}
\begin{field}
Числовым рядом называется выражение вида $a_1 + a_2 + a_3 + ...+ a_n + \ldots =  \sum\limits_{n=1}^{\infty}a_{n}$. Где \{$a_n$\} - заданная числовая последовательность.
Частичной суммой $S_{n}$ ряда \{$a_{n}$\}  называется сумма $a_{1} + a_{2} + a_{3} + ... + a_{n}$. Если существует конечный предел $\lim\limits_{n \to \infty} S_{n} = S$, то говорят, что ряд сходится и S называют суммой ряда.
\end{field}
\end{note}

\begin{note}
\begin{field}
Пример сходящегося ряда: геометрическая прогрессия.
\end{field}
\begin{field}
Дана геометрическая прогрессия $\sum\limits_{n=1}^{\infty}aq^n$. Частичная сумма такого ряда $S_n = a_1 + aq + aq^2 + ... + aq^{n-1}$. Если |q| < 1, то существует предел, $\lim\limits_{n \to \infty}S_n = \lim\limits_{n \to \infty}\frac{1 - q^n}{1 - q}a = \frac{a}{1 - q}$. Мы видим, что геометрическая прогрессия со знаменателем q при |q| < 1 дает пример сходящегося ряда, причем  $\sum_{n=1}^{\infty}aq^n = \frac{a}{1 - q}$.

\end{field}
\end{note}

\begin{note}
\begin{field}
Критерий Коши сходимости числового ряда.
\end{field}
\begin{field}
Ряд  $\sum\limits_{n=1}^{\infty}a_n$ сходится тогда и только тогда, когда для любого $\varepsilon$ > 0 существует число N, такое, что при n $\geq$ N и при p = 1, 2, 3... выполняется неравенство $\mid  \sum\limits_{k=n+1}^{n + p}a_n \mid < \varepsilon$

\end{field}
\end{note}

\begin{note}
\begin{field}
Доказательство расходимости гармонического ряда с помощью критерия Коши.
\end{field}
\begin{field}
Рассмотрим гармоничкский ряд $\sum\limits_{n=1}^{\infty}\frac{1}{n}$. Пусть $\varepsilon = \frac{1}{2}$. Найдем N, о котором идет речь в критерии Коши, и возьмем n $\geq$ N, p = n. Если бы критерий Коши выполнялся для этого ряда, то было бы верно неравенство: $\frac{1}{n + 1} + \frac{1}{n + 2} + ... + \frac{1}{2n} < \frac{1}{2}$, с другой стороны $\frac{1}{n + 1} + \frac{1}{n + 2} + ... + \frac{1}{2n} \geq n\frac{1}{2n} = \frac{1}{2}$. Из двух последних неравенств получим $\frac{1}{2} < \frac{1}{2}$. Это противоречие показывает, что требование кр. Коши для гармонического ряда не выполнены.


\end{field}
\end{note}

\begin{note}
\begin{field}
Необходимое условие сходимости и замечание.
\end{field}
\begin{field}
\begin{thm}
	Если ряд $\sum\limits_{n=1}^{\infty}a_{n}$ cходится, то $\lim\limits_{n \to \infty}a_{n} = 0$.
\end{thm}
Замечание. Условие $\lim\limits_{n \to \infty}a_n = 0$ не является достаточным условием сходимости числового ряда. Это условие выполняется для гармонического ряда, но этот ряд расходится.


\end{field}
\end{note}

\begin{note}
\begin{field}
Доказательство расходимости геометрической прогрессии с помощью необходимого условия сходимости.
\end{field}
\begin{field}
Рассмотрим ряд $\sum\limits_{n=0}^{\infty}aq^n$ при a $\neq$ 0, |q| > 1. В этом случае получаем $\lim\limits_{n \to \infty}aq^n = \infty$. Необходимое условие сходимости нарушется $\Longrightarrow$ ряд расходится.

\end{field}
\end{note}

\begin{note}
\begin{field}
Признак сравнения сходимости положительного ряда.
\end{field}
\begin{field}
Пусть даны два ряда $\sum\limits_{n=1}^{\infty}a_n$ и $\sum\limits_{n=1}^{\infty}b_n$, причем $0 	\leq a_{n} \leq b_{n}$, n = 1, 2, 3, \ldots Тогда из сходимости второго следует сходимость первого, а из расходимости первого следует расходимость второго.

\end{field}
\end{note}

\begin{note}
\begin{field}
Признак сравнения сходимости положительного ряда в предельной форме. Замечание об использовании эквивалентных бесконечно малых.
\end{field}
\begin{field}
Пусть даны два ряда $\sum\limits_{n=1}^{\infty}a_{n}$ и $\sum\limits_{n=1}^{\infty}b_n$, и пусть $a_{n} > 0$, $b_{n} > 0$, n = 1, 2, 3, \ldots Тогда если существует предел $\lim\limits_{n \to \infty} \frac{a_{n}}{b_{n}} = k$, $0 < k < +\infty$, то эти ряды сходятся или раходсятся одновременно.
\newline
Замечание. Если в условиях предыдущей теоремы $a_n$ и $b_n$ - бесконечно малые последовательности, и $a_n \sim b_n$ при $n \to \infty$, то утверждение этой теоремы остается в силе.


\end{field}
\end{note}

\begin{note}
\begin{field}
Признак Даламбера и радикальный признак Коши сходимости положительного ряда.
\end{field}
\begin{field}


Признак Даламбера. Пусть ряд  $\sum\limits_{n=0}^{\infty}a_n$ составлен из положительных чисел и пусть существует предел $\lim\limits_{n \to \infty} \frac{a_{n + 1}}{a_{n}} = q$. Тогда, если q < 1, то ряд сходится, а если q > 1, то расходится. При q = 1 может как сходиться, так и расходиться.
\newline
Радикальный признак Коши. Пусть ряд $\sum\limits_{n=1}^{\infty}a_{n}$ составлен из положительных чисел, и пусть существует предел  $\lim\limits_{n \to \infty} \sqrt[n]{a_{n}} = q$. Тогда, если q < 1, то ряд сходится, а если q > 1, то расходится. При q = 1 может как сходиться, так и расходиться.


\end{field}
\end{note}

\begin{note}
\begin{field}
Интегральный признак Коши и исследование с его помощью сходимости обобщённого гармонического ряда
\end{field}
\begin{field}
Пусть функция $f : [0, \infty] \to \mathbb{R}$ положительна, непрерывна и монотонна. Тогда ряд $\sum\limits_{n=1}^{\infty}f(n)$ и интеграл $\int\limits_1^\infty f(x) dx$ сходятся или расходятся одновременно.
\newline
Обобщенный гармонический ряд $\sum\limits_{n=1}^{\infty}\frac{1}{n^s}$. Функция $f(x) = \frac{1}{x^s}$ на промежутке $[1; +\infty)$ удовлетворяет требованиям интегрального признака Коши. Поэтому обобщенный гармонический ряд сходится или расходится одновременно с интергралом $\int\limits_{1}^{\infty}\frac{dx}{x^s}$. Пусть $s \neq 1$. Тогда $\int\limits_{1}^{\infty} = \frac{x^{1 - s}}{1- s}\bigg|_1^{\infty}$. Ясно, что при s > 1 интеграл сходится, а при s < 1 интеграл расходится. Если s = 1, то $\int\limits_{1}^{\infty}\frac{dx}{x} = ln(x)\bigg|_1^{\infty}$. Отсюда следует, что ряд $\sum\limits_{n=1}^{\infty}$ cходится при s > 1, и расходится при при $s \leq 1$.



\end{field}
\end{note}

\begin{note}
\begin{field}
Абсолютная сходимость. Теорема о сходимости абсолютно сходящегося ряда.
\end{field}
\begin{field}
Пусть дан ряд  $\sum\limits_{n=1}^{\infty}a_n$. Если при этом сходится ряд $\sum\limits_{n=1}^{\infty}\mid a_n \mid$, то исходный ряд называется абсолютно сходящимся. Если же ряд $\sum\limits_{n=1}^{\infty}a_n$ cходится, а ряд  $\sum\limits_{n=1}^{\infty}\mid a_n \mid$ расходится, то исходный ряд называется условно сходящимся.
\begin{thm}
	Абсолютно сходящийся ряд сходится.
\end{thm}
\begin{proof}
	Т.к. ряд $\sum\limits_{n=1}^{\infty}|a_n|$ сходится, то для него выполнены требования критерия Коши. Это означает, что $\forall \varepsilon > 0$ $\exists N : \forall n \geq N, p = 1, 2, 3, ...$  $\bigg| \sum\limits_{k=n+1}^{n + p}|a_k| \bigg| < \varepsilon$. Далее $\bigg| \sum\limits_{k=n+1}^{n + p}|a_k| \bigg|$ = $\sum\limits_{k=n+1}^{n + p}|a_k| $ $ \geq $ $ \bigg| \sum\limits_{k=n+1}^{n + p}a_k \bigg| $. Мы видим, что для указанных $n$ и $p$ выполняется неравенство $\bigg| \sum\limits_{k=n+1}^{n + p}a_k \bigg| < \varepsilon$. Это означает, что требования критерия Коши выполнены для ряда $\sum\limits_{n=1}^{\infty}a_n$. Используя достаточность этих требований для сходимости ряда делаем отсюда вывод о том, что ряд $\sum\limits_{n=1}^{\infty}a_n$ cходится.
\end{proof}
\end{field}
\end{note}

\begin{note}
\begin{field}
Ряд Лейбница и теорема Лейбница. Пример условно сходящегося ряда.
\end{field}
\begin{field}
Ряд  $\sum_{n=1}^{\infty}(-1)^{n-1}a_n$ называется рядом Лейбница, если $a_n > 0$, $a_n > a_{n+1}, n = 1, 2, 3...$ и
$\lim\limits_{n \to \infty}a_n = 0$.
\begin{thm}
Ряд Лейбница сходится
\end{thm}
Пример. Ряд Лейбница $\sum\limits_{n=1}^{\infty}\frac{(-1)^{n - 1}}{n}$ cходится условно, а ряд  $\sum\limits_{n=1}^{\infty}\frac{(-1)^{n - 1}}{n^2}$ сходится абсолютно.


\end{field}
\end{note}

\begin{note}
\begin{field}
Теорема об оценке остатка ряда Лейбница.
\end{field}
\begin{field}
\begin{thm}
	Пусть дан ряд Лейбница $\sum\limits_{n=1}^{\infty}\left( -1 \right)^{n-1}a_n $, и пусть $S$ - сумма этого ряда, а $S_N$ - сумма его первых N членов. Тогда $0 < |S - S_N| < a_{N + 1}$
\end{thm}


\end{field}
\end{note}

\begin{note}
\begin{field}
Определение функциональной последовательности и поточечной сходимости.
\end{field}
\begin{field}

Пусть имеется последовательность $\{f_n(x)\}$, элементами которой являются вещественнозначные функции, заданные на одном и том же множестве $X$. Если вместо $x$ подставить конкретный $x_0 \in X$, то мы получим числовую последовательность. Если эта последовательность сходится, то ее предел обозначается $f(x_0)$. Рассмотрим множество $X_0 \subset X$, которое состоит из элементов множества $X$, при которых определена функция $f$ (т.е. при которых сходится соответствуящая числовая последовательность). Говорят что на множестве $X_0$ последовательность $\{f_n(x)\}$ поточечно сходится к функции $f(x)$

\end{field}
\end{note}

\begin{note}
\begin{field}
Определение метрического пространства и сходимости в нём
\end{field}
\begin{field}
Рассмотрим определение метрического просранства. Пусть $X$ - некоторое множество, и пусть каждой паре $(x, y)$ поставлено в соответствие вещественное число $\rho(x, y)$ и при этом выполнены следующие условия.


\begin{enumerate}
	\item{$\rho(x, y) \geq 0$, $\rho(x, y) = 0 \iff x = y$}
	\item{$\rho(x, y) = \rho(y, x)$}
	\item{$\rho(x, y) \leq \rho(x, z) + \rho(y, z)$ - неравентсво треугольника.}
\end{enumerate}
В метрическом пространстве можно ввести понятие сходимости последовательности элементов этого пространства. Пуст $X$ - метрическое пространство пусть $\{x_n\}_{n=1}^{\infty}$ - последовательность элементов этого пространства. Говорят, что $\{x_n\}$ сходится к $x$, если $\lim\limits_{n \to \infty}\rho(x_n, x) = 0$.


\end{field}
\end{note}

\begin{note}
\begin{field}
Норма в линейном пространстве и сходимости по норме.
\end{field}
\begin{field}
Пусть $Z$ - линейное пространство. Это пространство можно называется нормированным, если каждому $x \in Z$ поставлено в соответствие число $\|x\|$ - норма этого элемента и при этом выполнены основные свойства нормы

\begin{enumerate}
	\item{$\|x\| \geq 0$, $\|x\| = 0 \iff x = 0$}
	\item{$\|\alpha x\| = \mid\alpha\mid\|x\|$}
	\item{$\|x + y\| \leq \|x\| + \|y\|$ - неравентсво треугольника.}
\end{enumerate}


\end{field}
\end{note}

\begin{note}
\begin{field}
Норма в пространстве непрерывных на отрезке функций; норма в пространстве ограниченных функций
\end{field}
\begin{field}
Пусть $Z$ - пространство всех непрерывных на $[a, b]$ функций. Норму элемента $f \in Z$ определим так $\|f\|$ =
$\max\limits_{x \in [a, b]}|f(x)|$
\newline
Можно рассмотреть так же и множество всех ограниченных на отрезке $[a, b]$ функций. Здесь ввести норму элемента $f$ можно по такому правилу: $\|f\| = \sup\limits_{x \in[a, b]}|f(x)|$.


\end{field}
\end{note}

\begin{note}
\begin{field}
Определение равномерной сходимости функциональной последовательности.
\end{field}
\begin{field}
Пусть функциональная последовательность $\{f_n(x)\}$ поточечно сходится на множестве $X$ к функции $f(x)$. Эта сходимость называется равномерной, если $\forall \varepsilon > 0$ $\exists N$ : $\forall n \geq N$, $\forall x \in X \Longrightarrow |f_n(x) - f(x)| < \varepsilon$


\end{field}
\end{note}

\begin{note}
\begin{field}
Критерий Коши равномерной сходимости функциональной последовательности.
\end{field}
\begin{field}
\begin{thm}
	Пусть дана функциональная последовательность $\{f_n(x)\}$ cоставленная из функций, заданных на множестве $X$. Такая последовательность сходится равномерно на $X$ тогда и только тогда, когда $\forall \varepsilon > 0$ $\exists N$ : $\forall n \geq N$, p = 1, 2, 3, ... $\Longrightarrow \forall x \in X$ $|f_n(x) - f_{n + p}(x)| < \varepsilon$

\end{thm}


\end{field}
\end{note}

\begin{note}
\begin{field}
Необходимое и достаточное условие равномерной сходимости функциональной последовательности.
\end{field}
\begin{field}
Последовательность $\{f_n(x)\}$, составленная из функций, заданных на множестве $X$, cходится к функции $f:X \to \mathbb{R} \Longleftrightarrow \lim\limits_{n \to \infty}\sup\limits_{x \in X}|f_n(x) - f(x)| = 0$


\end{field}
\end{note}

\begin{note}
\begin{field}
Определение функционального ряда. Поточечная и равномерная сходимость.
\end{field}
\begin{field}
Пусть имеется функциональная последовательность $\{f_n(x)\}$, элементами которой являются функции, заданные на некотором множестве $X$. Мы можем рассмотреть функциональный ряд: $f_1(x) + f_2(x) + ... + f_n(x) + ...$ = $\sum\limits_{n=1}^{\infty}f_n(x)$.
\newline
Такой ряд называется поточечно сходящимся на множестве $X$ к функции $f : X \to \mathbb{R}$, если соответствующий числовой ряд при каждой фиксированной точке $x_0 \in X$ сходится к $f(x_0)$, $\sum\limits_{n=1}^{\infty}f_n(x_0)$ = $f(x_0)$
\newline
Поточечная сходимость ряда $\sum\limits_{n=1}^{\infty}f_n(x)$ к функции $f(x)$ на множестве $X$ называется равномерной, если $\forall \varepsilon > 0$ $\exists N$ : $\forall n \geq N$ $ \Longrightarrow \forall x \in X$ $|\sum\limits_{k=1}^{n}f_k(x) - f(x)| < \varepsilon$

\end{field}
\end{note}

\begin{note}
\begin{field}
Критерий Коши. Замечание об отбрасывании и добавление конечного числа членов ряда.
\end{field}
\begin{field}
Ряд $\sum\limits_{n=1}^{\infty}f_n(x)$ сходится равномерно на множестве $X$ тогда и только тогда, когда $\forall \varepsilon$ $\exists N = N(\varepsilon)$ : $n \geq N$, $p = 1, 2, 3, ... \Longrightarrow \mid\sum\limits_{k=n+1}^{n+p}f_k(x) \mid$ < $\varepsilon$
\newline
Замечание. Пусть дан функциональный ряд $\sum\limits_{k=1}^{\infty}f_k(x)$. Остатком такого ряда называется ряд $\sum\limits_{k=n+1}^{\infty}f_k(x)$. Если ряд сходится на множестве $X$ (поточечно или равномерно), то сходится и любой из его остатков, соответственно поточечно или равномерно. Если сходится хотя бы один остаток, то сходится и сам ряд (соответственно поточечно или равномерно).
\newline
С помощью критерия Коши можно доказать, что отбрасывание (или приписывание) конечного числа членов не отражается на сходимости (или расходимости) ряда, а также на характере сходимости. Это замечание относится и к числовым рядам.

\end{field}
\end{note}

\begin{note}
\begin{field}
Теорема о непрерывности суммы функционального ряда.
\end{field}
\begin{field}
\begin{thm}
	Пусть дан ряд $\sum\limits_{k=1}^{\infty}f_k(x)$, составленый из непрерывных на множестве $[a, b]$ функций, и пусть на указанном отрезке этот ряд равномерно сходится к функции $f(x)$. Тогда функция $f(x)$ непрерывна на $[a, b]$
\end{thm}

\end{field}
\end{note}

\begin{note}
\begin{field}
Теорема о почленном интегрировании функционального ряда.
\end{field}
\begin{field}
\begin{thm}
Пусть ряд $\sum\limits_{k=1}^{\infty}f_k(x)$ cоставлен из непрерывных на отрезке $[a, b]$ функций и пусть этот ряд равномерно на $[a, b]$ сходится к функции $f(x)$. Тогда $\int\limits_{a}^{b}f(x)dx$ = $\sum\limits_{k=1}^{\infty}\int\limits_{a}^{b}f_k(x)dx$
\end{thm}

\end{field}
\end{note}

\begin{note}
\begin{field}
Теорема о почленном дифференцировании функционального ряда.
\end{field}
\begin{field}
\begin{thm}
Пусть дан ряд $\sum\limits_{k=1}^{\infty}f_k(x)$, составленый из непрерывно дифференцируемых на интервале $(a, b)$ функций, который сходится на этом интервале поточечно к функции $f(x)$. Пусть далее ряд $\sum\limits_{k=1}^{\infty}f'_k(x)$ сходится к своей сумме равномерно на интервале $(a, b)$. Тогда справдливо равенство $f'(x)$ = $\sum\limits_{k=1}^{\infty}f'_k(x)$
\end{thm}


\end{field}
\end{note}

\begin{note}
\begin{field}
Признак Вейерштрасса равномерной сходимости функционального ряда.
\end{field}
\begin{field}
\begin{thm}
	Пусть дан ряд $\sum\limits_{n=1}^{\infty}f_n(x)$, составленный из функций, определенных на множестве $X$. Предположим так же, что $\forall x \in X$ и $n = 1, 2, 3, ...$ выполняется неравенство $\mid f_n(x) \mid \leq a_n$. Тогда если числовой ряд $\sum\limits_{n=1}^{\infty}a_n$ сходится, то исходный ряд $\sum\limits_{n=1}^{\infty}f_n(x)$ сходится абсолютно и равномерно на множестве $X$.
\end{thm}

\end{field}
\end{note}

\begin{note}
\begin{field}
Признак Дирихле равномерной сходимости функционального ряда.
\end{field}
\begin{field}
\begin{thm}
	Пусть имеется ряд $\sum\limits_{n=1}^{\infty}a_n(x)b_n(x)$ причем функции $a_n(x)$ и $b_n(x)$ определены на множестве $X$. Такой ряд сходится равномерно на $X$, если выполенены условия:
	\begin{enumerate}
		\item{Частичные суммы ряда $\sum\limits_{n=1}^{\infty}b_n(x)$ равномерно ограничены, т.е.
		\[
			\mid\sum\limits_{k=1}^{n}b_k(x)\mid \leq M \forall x \in X \text{и} \forall n = 1, 2, ...
		\] }
		\item{Последовательность функций $\{a_n(x)\}$ не возрастает (т.е. $\forall x \in X$ выполняется неравенство $a_n(x)  \geq  a_{n+1}(x), n = 1, 2, 3, ...$) и равномерно на множестве $X$ cходится к нулю (т.е к тождественно равной нулю функции)}
	\end{enumerate}
\end{thm}


\end{field}
\end{note}

\begin{note}
\begin{field}
Признак Абеля равномерной сходимости функционального ряда.
\end{field}
\begin{field}
\begin{thm}
	Пусть имеется ряд $\sum\limits_{n=1}^{\infty}a_n(x)b_n(x)$, составленный из  функций заданных на множестве $X$. Этот ряд сходится равномерно на $X$, если выполенены условия:
	\begin{enumerate}
		\item{Ряд $\sum\limits_{n=1}^{\infty}b_n(x)$ сходится на множестве $X$ равномерно.}
		\item{Последовательность $\{a_n(x)\}$ не возрастает, т.е. $a_n(x) \geq a_{n + 1}(x)$ при любом $x \in X$, и равномерно ограничена: $\mid a_n(x) \mid$ $\leq$ $M$ $\forall x \in X$ и $\forall n=1,2,3,...$  }
	\end{enumerate}

\end{thm}


\end{field}
\end{note}

\begin{note}
\begin{field}
Первая теорема Абеля о степенных рядах.
\end{field}
\begin{field}
\begin{thm}
	Пусть ряд $\sum\limits_{n=0}^{\infty}a_n(x-x_0)^n$ сходится при некотором $x = x_1$, $x_1 \neq x_0$. Тогда этот ряд сходится абсолютно и равномерно на отрезке $[x_0 - k, x_0 + k]$, где $k$ любое положительное число меньшее $\mid x_1 - x_0 \mid$
\end{thm}


\end{field}
\end{note}

\begin{note}
\begin{field}
Определение радиуса и интервала сходимости. Характер сходимости степенного ряда.
\end{field}
\begin{field}
Пуст $A$ - множество тех $x$, при которых ряд $\sum\limits_{n=0}^{\infty}a_n(x-x_0)^n$ сходится.
\newline
Определим теперь радиус сходимости степенного ряда $\sum\limits_{n=0}^{\infty}a_n(x-x_0)^n$ равенством $R = \sup\limits_{x \in A}\mid x - x_0 \mid$. Такое $R$ всегда существует, т.к $A$ не пусто. $R$ есть неотрицательное вещественное число или $+\infty$.
\newline
Интервал $(x_0 - R, x_0 + R)$ называется интервалом сходимости степенного ряда $\sum\limits_{n=0}^{\infty}a_n(x-x_0)^n$
\newline
Если $0 < k < R$, то степенной ряд $\sum\limits_{n=0}^{\infty}a_n(x-x_0)^n$ сходится равномерно и абсолютно на отрезке $[x_0 - k, x_0 + k]$. На всем интервале сходимости равномерной сходимости может и не быть (такая сходимость называется равномерной сходимостью внутри интервала). На концах интервала сходимости $x_0 \pm R$ степенной ряд может вести себя по-разному. Вне отрезка $[x_0 - R, x_0 + R]$ степенной ряд расходится.

\end{field}
\end{note}

\begin{note}
\begin{field}
Формула Коши-Адамара.
\end{field}
\begin{field}
Для вычисления радиуса сходимости степенного ряда $\sum\limits_{n=0}^{\infty}a_n(x-x_0)^n$ можно использовать формулу Коши-Адамара $\frac{1}{R} = \varlimsup\limits_{n \to \infty}\sqrt[n]{|a_n|}$


\end{field}
\end{note}

\begin{note}
\begin{field}
Вторая теорема Абеля о степенных рядах.
\end{field}
\begin{field}
\begin{thm}
	Пусть $R$ - радиус сходимости степенного ряда $\sum\limits_{n=0}^{\infty}a_n(x-x_0)^n$,  $0 < R < +\infty$. Предположим, что этот ряд сходится при $x = x_0 + R$. Тогда этот ряд сходится равномерно на отрезке $[x_0, x_0 + R]$
\end{thm}

\end{field}
\end{note}

\begin{note}
\begin{field}
Лемма о радиусах сходимости почленно продифференцированного и проинтегрированного степенного ряда.
\end{field}
\begin{field}
Пусть $R > 0$ - радиус сходимости степенного ряда $\sum\limits_{n=0}^{\infty}a_n(x-x_0)^n$. Тогда радусы сходимости рядов $\sum\limits_{n=0}^{\infty}na_n(x-x_0)^{n-1}$ и $\sum\limits_{n=0}^{\infty}\frac{a_{n}(x-x_0)^{n+1}}{n+1}$ равны $R$.



\end{field}
\end{note}

\begin{note}
\begin{field}
Теорема о почленном интегрировании степенного ряда.
\end{field}
\begin{field}
\begin{thm}
	Пусть дан ряд  $\sum\limits_{n=1}^{\infty} a_n \left( x - x_0 \right)^n $ и пусть $R > 0$ - его радиус сходимости. Тогда этот ряд допускает почленное интегрирование по любому отрезку  $[\alpha, \beta] \subset \left( x_0 - R, x_0 + R \right) $.
\end{thm}


\end{field}
\end{note}

\begin{note}
\begin{field}
Теорема о почленном дифференцировании степенного ряда.
\end{field}
\begin{field}
\begin{thm}
	Пусть дан ряд  $\sum\limits_{n=1}^{\infty} a_n \left( x - x_0 \right)^n $ и пусть $R > 0$ - его радиус сходимости. Тогда при почленном дифференцировании радиус сходимости не меняется. В каждой точке интервала сходимости $(x_0 - R, x_0 + R)$ сумма степенного ряда бесконечно дифференцируема и производная этой суммы $f^{(k)}(x)$ может быть вычислена с помощью почленного дифференцирования (соответсвующие число раз):
	\[
		f^{\left(k\right) }\left(x\right) = \sum\limits_{n=k}^{\infty} a_n n(n-1)\ldots\left(n - k +1\right)(x - x_0)^{n - k}, k =1, 2, 3, \ldots
	.\]
\end{thm}


\end{field}
\end{note}

\begin{note}
\begin{field}
Ряд Тейлора. Пример функции, которая не представляется суммой своего ряда Тейлора.
\end{field}
\begin{field}
Пусть функция $f(x)$ имеет в точке  $x_0$ производные всех порядков.
\newline
Ряд $\sum\limits_{n=0}^{\infty} \frac{f^{(n)}(x_0)(x - x_0)^n}{n!}$ называется рядом Тейлора.
\newline
Рассмотрим функцию
\[
	\varphi\left( x \right) =
	\begin{cases}
		e^{-\frac{1}{x^2}}, & x \neq  0 \\
		0, & x \neq  0
	\end{cases}
.\]
Можно доказать, что при $x = 0$ эта функция имеет производные всех порядков, и все эти производные равны нулю. Ряд тейлора в данном случае состоит из нулей, и он воспроизводит функцию  $\varphi$ лишь при $x = 0$


\end{field}
\end{note}

\begin{note}
\begin{field}
Разложения в ряды по степеням x основных элементарных функций (пять разложений).
\end{field}
\begin{field}
\begin{enumerate}
	\item $e^x = \sum\limits_{n=0}^{\infty} \frac{x^n}{n!}$, $-\infty < x < +\infty$
	\item $\cos\left( x \right) = \sum\limits_{n=0}^{\infty}(-1)^n \frac{x^{2n}}{2n!} $, $-\infty < x < +\infty$
	\item $\sin(x) = \sum\limits_{n=0}^{\infty} (-1)^n \frac{x^{2n + 1}}{(2n + 1)!}$, $-\infty < x < +\infty$
	\item $\ln\left( 1 + x \right)  = \sum\limits_{n=1}^{\infty} (-1)^{n-1} \frac{x^n}{n}$, $-1 < x < 1$
	\item  $ (1 + x)^\alpha = 1 + \sum\limits_{n=1}^{\infty} \frac{\alpha(\alpha - 1)\ldots(\alpha - n + 1)}{n!}x^n$, $ |x| < 1 $
\end{enumerate}


\end{field}
\end{note}

\begin{note}
\begin{field}
Определение собственного интеграла, зависящего от параметра.
\end{field}
\begin{field}
Пусть функция $f(x, y)$ определена для  $x \in [a, b]$ и  $y \in Y$. Предположим также, что  $\forall y \in Y$ существует (в собственном смысле) интеграл $I(y) = \int\limits_{a}^{b}f\left(x, y \right)dx$. Такой интеграл называется (собственным) интегралом, зависящим от параметра.
\end{field}
\end{note}


\end{document}
